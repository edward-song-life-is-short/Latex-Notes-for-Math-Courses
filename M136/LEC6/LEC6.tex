\documentclass[11pt]{article}

\usepackage{amsmath, amsfonts, amssymb, amsthm}
\usepackage{braket}
\usepackage{fullpage}
\usepackage[top=2cm, bottom=4.5cm, left=2.5cm, right=2.5cm]{geometry}
\usepackage{bbold}
\usepackage{enumitem}
\usepackage{mathtools}

\DeclarePairedDelimiter\ceil{\lceil}{\rceil}
\DeclarePairedDelimiter\floor{\lfloor}{\rfloor}

\usepackage{fancyhdr}
\usepackage{mathrsfs}
\usepackage{xcolor}

\usepackage{listings}
\usepackage{hyperref}
\usepackage[T1]{fontenc}
\usepackage{tabularx}

\usepackage{mathpazo}
\usepackage{xcolor}
\usepackage{float}

\usepackage{graphicx}
\usepackage{subcaption}
\usepackage[export]{adjustbox}
\usepackage{wrapfig}


\usepackage{tikz,lipsum,lmodern}
\usetikzlibrary{calc}
\usetikzlibrary{arrows}
\usepackage{pgfplots}
\usepackage{graphicx}
\makeatletter
\usepackage[most]{tcolorbox}

\setlength{\parindent}{0pt} 
\theoremstyle{plain}
\newtheorem*{theorem}{Theorem}

\theoremstyle{remark}
\newtheorem*{solution}{Solution}

\theoremstyle{plain}
\newtheorem*{claim}{Claim}

\newcolumntype{P}[1]{>{\centering\arraybackslash}p{#1}}
\pagestyle{fancy}
% \newcommand{\ehx}[]{} set new commands
\newcommand{\bd}{\textbf}
\newcommand\course{CSE 3500}

\newcommand{\fa}{\forall}

\newcommand{\nn}{\mathbb{N}}
\newcommand{\z}{\mathbb{Z}}
\newcommand{\rn}{\mathbb{R}}
\newcommand{\stopIndent}{\noindent\underline\bd}


\newcommand*{\rom}[1]{\expandafter\@slowromancap\romannumeral #1@}
\headheight 35pt
\rhead{Edward Song \\ \today \\ Math 136}
\title{Math 136}
\author{Edward Song}

\headsep 1.8em
\lfoot{}
\cfoot{}
\rfoot{\small\thepage}

\newtcbtheorem[auto counter,number within=section]{theo}%
  {Theorem}{fonttitle=\bfseries\upshape, fontupper=\slshape,
     arc=0mm, colback=blue!5!white,colframe=blue!75!black}{theorem}

\begin{document}

\section*{2 Spans, Sections, Lines}


% \begin{tcolorbox}[colback=red!5!white,colframe=red!75!black,title=Recall]
   
% \end{tcolorbox}   

\subsection*{Lines and Planes in $\rn^n$}

\begin{tcolorbox}[colback=green!5!white,colframe=green!75!black,title=Definition: Recall]
  A line in $\rn^n$ through the origin is a set of the form:
  \[\mathbb{L}=\{0+\overrightarrow{d}:t\in \rn\}=span\{\overrightarrow{d}\}\]
  where $\overrightarrow{d}\in\rn^n$ is non-zero. This is a direction vector.
\end{tcolorbox}   

\begin{tcolorbox}[colback=green!5!white,colframe=green!75!black,title=Example ]
    A line in $\rn^n$ is a set of the form: \[l=\{\overrightarrow{p}+t\overrightarrow{d}:t\in\rn\}\]
    where $\overrightarrow{p},\overrightarrow{d}\in\rn^n$ and $\overrightarrow{d}\neq \overrightarrow{0}$.
\end{tcolorbox}   

\begin{tcolorbox}[colback=magenta!5!white,colframe=magenta!75!black,title=Example ]
    \[l=\{\begin{bmatrix}
        2\\1
    \end{bmatrix}\}+t\begin{bmatrix}
        1\\2
    \end{bmatrix}:t\in\rn\]
\end{tcolorbox}   

\bd{Convention:} $\overrightarrow{v}=\begin{bmatrix}
    a\\b
\end{bmatrix}$ is on $l$ if its terminal part $(a,b)$ is on $l$. So for example, $\begin{bmatrix}
    1\\2
\end{bmatrix}$ and $\begin{bmatrix}
    2\\3
\end{bmatrix}$ are on $l$ $\iff (1,2)$ and $(2,3)$ are on $l$. 


\begin{tcolorbox}[colback=magenta!5!white,colframe=magenta!75!black,title=Problem 4]
   \bd{True or False:} The lines $l_1=\{\begin{bmatrix}
       1\\0\\0
   \end{bmatrix}+t\begin{bmatrix}
       1\\-1\\2
   \end{bmatrix}:t\in\rn\}$, $l_2=\{\begin{bmatrix}
       1\\3\\-4
   \end{bmatrix}+t\begin{bmatrix}
       2\\0\\-5
   \end{bmatrix}:t\in\rn\}$ are the same. Graph it and you see it does not work.\\

   Algebraically:
   $\begin{bmatrix}
       1\\0\\0
   \end{bmatrix}\in l_1$. Is $\begin{bmatrix}
       1\\0\\0
   \end{bmatrix}\in l_2 \iff$ can 1 solve for $t$ such that $\begin{bmatrix}
       1\\0\\0
   \end{bmatrix}=\begin{bmatrix}
       1\\3\\-4
   \end{bmatrix}+\begin{bmatrix}
       2\\0\\-5
   \end{bmatrix}$
\end{tcolorbox}   

\begin{tcolorbox}[colback=magenta!5!white,colframe=magenta!75!black,title=Problem 4]
    \bd{True or False:} The lines $l_1=\{\begin{bmatrix}
        1\\0
    \end{bmatrix}+t\begin{bmatrix}
        1\\-1
    \end{bmatrix}:t\in\rn\}$, $l_2=\{\begin{bmatrix}
        -1\\2
    \end{bmatrix}+t\begin{bmatrix}
        -2\\2\\
    \end{bmatrix}:t\in\rn\}$ are the same.
 
    Algebraically:
    Suppose $\overrightarrow{d_1}\begin{bmatrix}
        1\\-1
    \end{bmatrix}$ and $\overrightarrow{d_2}=\begin{bmatrix}
        -2\\2
    \end{bmatrix}$

    $\overrightarrow{d_2}=)-2\overrightarrow{d_1}$ are parallel. Now we need to check 
    if they intersect.
    
    \begin{align*}
        \begin{bmatrix}
            1\\0
        \end{bmatrix}&=\begin{bmatrix}
            -1\\2
        \end{bmatrix}+t\begin{bmatrix}
            -2\\2
        \end{bmatrix}\\
        \begin{bmatrix}
            1\\0
        \end{bmatrix}&=\begin{bmatrix}
            -1-2t\\2+2t
        \end{bmatrix}\\
        1&=-1-2t\\
        0&=2+2t\\
        t&=-1
    \end{align*}

    $\therefore\begin{bmatrix}
        1\\0
    \end{bmatrix}$ is on $l_2$ and they intersect.
 \end{tcolorbox}  

\begin{tcolorbox}[colback=green!5!white,colframe=green!75!black,title= Definition]
    A plane in $\rn^n$ through the origin is a set of the form:
    \[p=\{s\overrightarrow{u}+t\overrightarrow{v}:s,t\in\rn\}=span\{\overrightarrow{u}, \overrightarrow{v}\}\]
    where $\overrightarrow{u}, \overrightarrow{v}\in\rn^n, \overrightarrow{u}\neq c\overrightarrow{v}$ for only $c\in\rn$.
\end{tcolorbox}   

\begin{tcolorbox}[colback=magenta!5!white,colframe=magenta!75!black,title= Example]
    \[span\{\begin{bmatrix}
        1\\0\\0
    \end{bmatrix}\}, \begin{bmatrix}
        1\\0\\0
    \end{bmatrix}=p\]
\end{tcolorbox}   

\begin{tcolorbox}[colback=green!5!white,colframe=green!75!black,title= Definition]
    A plane in $\rn^n$ is a set of the form:
    \[p=\{\overrightarrow{w}+s\overrightarrow{u}+t\overrightarrow{v}:s,t\in\rn\}\]
    where $\overrightarrow{w},\overrightarrow{u},\overrightarrow{v}\in\rn^n$ and $\overrightarrow{u}\neq c\overrightarrow{v}$ for any $c\in\rn$.
\end{tcolorbox} 

\begin{tcolorbox}[colback=magenta!5!white,colframe=magenta!75!black,title= Example]
    \[p=\{\begin{bmatrix}
        1\\2\\0
    \end{bmatrix}\}+s\begin{bmatrix}
        1\\0\\0
    \end{bmatrix}+t\begin{bmatrix}
        1\\0\\3
    \end{bmatrix}:s,t\in\rn\]
\end{tcolorbox}   

\end{document}