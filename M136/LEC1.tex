\documentclass[11pt]{article}

\usepackage{amsmath, amsfonts, amssymb, amsthm}
\usepackage{braket}
\usepackage{fullpage}
\usepackage[top=2cm, bottom=4.5cm, left=2.5cm, right=2.5cm]{geometry}
\usepackage{bbold}
\usepackage{enumitem}
\usepackage{mathtools}
\DeclarePairedDelimiter\ceil{\lceil}{\rceil}
\DeclarePairedDelimiter\floor{\lfloor}{\rfloor}

\usepackage{fancyhdr}
\usepackage{mathrsfs}
\usepackage{xcolor}
\usepackage{graphicx}
\usepackage{listings}
\usepackage{hyperref}
\usepackage[T1]{fontenc}
\usepackage{tabularx}

\usepackage{mathpazo}
\usepackage{xcolor}

\usepackage{tikz,lipsum,lmodern}
\usetikzlibrary{calc}
\usetikzlibrary{arrows}
\usepackage{pgfplots}

\makeatletter
\usepackage[most]{tcolorbox}

\setlength{\parindent}{0pt} 
\theoremstyle{plain}
\newtheorem*{theorem}{Theorem}

\theoremstyle{remark}
\newtheorem*{solution}{Solution}

\theoremstyle{plain}
\newtheorem*{claim}{Claim}

\newcolumntype{P}[1]{>{\centering\arraybackslash}p{#1}}
\pagestyle{fancy}
% \newcommand{\ehx}[]{} set new commands
\newcommand{\bd}{\textbf}
\newcommand\course{CSE 3500}

\newcommand{\fa}{\forall}

\newcommand{\nn}{\mathbb{N}}
\newcommand{\z}{\mathbb{Z}}
\newcommand{\rn}{\mathbb{R}}
\newcommand{\stopIndent}{\noindent\underline\bd}


\newcommand*{\rom}[1]{\expandafter\@slowromancap\romannumeral #1@}
\headheight 35pt
\rhead{Edward Song \\ \today \\ Math 136}
\title{Math 135}
\author{Edward Song}

\headsep 1.8em
\lfoot{}
\cfoot{}
\rfoot{\small\thepage}

\newtcbtheorem[auto counter,number within=section]{theo}%
  {Theorem}{fonttitle=\bfseries\upshape, fontupper=\slshape,
     arc=0mm, colback=blue!5!white,colframe=blue!75!black}{theorem}

\begin{document}

\section {Vectors in $\rn^n$}


% \begin{tcolorbox}[colback=red!5!white,colframe=red!75!black,title=Recall]
   
% \end{tcolorbox}   

\subsection{Introduction}

\begin{tcolorbox}[colback=green!5!white,colframe=green!75!black,title=Definition]
    A vector is an element of the set $\overrightarrow{x} = 
    \begin{bmatrix}
        x_1 \\ . \\ . \\ x_n
    \end{bmatrix}$
        
\end{tcolorbox}   

\subsubsection{Properties of Vectors}
\begin{enumerate}[label=\Roman*.]
    \item \bd{Equality:} $\overrightarrow{x} = \overrightarrow{u}$. If both belong to $\rn^n$, then $x_1 = u_1, x_n = u_n$.
    \item \bd{Addition:} $\overrightarrow{x} + \overrightarrow{u} = 
    \begin{bmatrix}
        v_1 + u_1 \\
        .
        \\
        v_n + u_n
    \end{bmatrix}$.
    \item \bd{Additive Inverse:} $\overrightarrow{x} + (-\overrightarrow{x}) = \overrightarrow{0}$
    \item \bd{Commutativity:} $\overrightarrow{x} + \overrightarrow{u} = \overrightarrow{u} + \overrightarrow{x}$.
    \item \bd{Associativity:} $\overrightarrow{x} + \overrightarrow{u} + \overrightarrow{w} = \overrightarrow{u} + (\overrightarrow{x} + \overrightarrow{w})$.
    \item \bd{Zero vector:} $\overrightarrow{0} = \begin{bmatrix}
        0 \\. \\. \\0
    \end{bmatrix}$
    \item \bd{Scalar Multiplication:} Let $c\in\rn$: $ c\cdot \overrightarrow{v} = \begin{bmatrix}
        cv_1\\.\\.\\cv_n
    \end{bmatrix}$
    \item \bd{Associativity of Multiplication:} Let $c,d \in \rn: (cd)\overrightarrow{v} = c(d\overrightarrow{v})$
    \item \bd{Distributive Property:} $c(\overrightarrow{x} + \overrightarrow{v}) = c\overrightarrow{x} + c\overrightarrow{v}$
    \item \bd{Inverse Property:} $0 \cdot \overrightarrow{v} = \overrightarrow{0}$
    \item If $c\overrightarrow{v} = \overrightarrow{0}$, then either $c = 0$ or $\overrightarrow{v} = \overrightarrow{0}$
    \end{enumerate}

\begin{tcolorbox}[colback=magenta!5!white,colframe = magenta!75!black,title=Example]
    \bd{Problem:} Prove \rom{11}

    \bd{Solution:} \\

    Left side:
    \begin{align*}
        c\overrightarrow{v} = \overrightarrow{0} \iff \begin{bmatrix}
            cv_1 \\ . \\cv_n
        \end{bmatrix} =  \begin{bmatrix}
            0 \\ . \\0
        \end{bmatrix} \\
    \end{align*}

    Right side:
    \begin{align*}
        c\overrightarrow{v} = \overrightarrow{0} \\
        \frac{1}{c}c\overrightarrow{v} = \frac{1}{c}\overrightarrow{0} \\
        c\overrightarrow{v} = \overrightarrow{0} \\
    \end{align*}

\end{tcolorbox}   

\end{document}

