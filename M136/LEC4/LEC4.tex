\documentclass[11pt]{article}

\usepackage{amsmath, amsfonts, amssymb, amsthm}
\usepackage{braket}
\usepackage{fullpage}
\usepackage[top=2cm, bottom=4.5cm, left=2.5cm, right=2.5cm]{geometry}
\usepackage{bbold}
\usepackage{enumitem}
\usepackage{mathtools}

\DeclarePairedDelimiter\ceil{\lceil}{\rceil}
\DeclarePairedDelimiter\floor{\lfloor}{\rfloor}

\usepackage{fancyhdr}
\usepackage{mathrsfs}
\usepackage{xcolor}

\usepackage{listings}
\usepackage{hyperref}
\usepackage[T1]{fontenc}
\usepackage{tabularx}

\usepackage{mathpazo}
\usepackage{xcolor}
\usepackage{float}

\usepackage{graphicx}
\usepackage{subcaption}
\usepackage[export]{adjustbox}
\usepackage{wrapfig}


\usepackage{tikz,lipsum,lmodern}
\usetikzlibrary{calc}
\usetikzlibrary{arrows}
\usepackage{pgfplots}
\usepackage{graphicx}
\makeatletter
\usepackage[most]{tcolorbox}

\setlength{\parindent}{0pt} 
\theoremstyle{plain}
\newtheorem*{theorem}{Theorem}

\theoremstyle{remark}
\newtheorem*{solution}{Solution}

\theoremstyle{plain}
\newtheorem*{claim}{Claim}

\newcolumntype{P}[1]{>{\centering\arraybackslash}p{#1}}
\pagestyle{fancy}
% \newcommand{\ehx}[]{} set new commands
\newcommand{\bd}{\textbf}
\newcommand\course{CSE 3500}

\newcommand{\fa}{\forall}

\newcommand{\nn}{\mathbb{N}}
\newcommand{\z}{\mathbb{Z}}
\newcommand{\rn}{\mathbb{R}}
\newcommand{\stopIndent}{\noindent\underline\bd}


\newcommand*{\rom}[1]{\expandafter\@slowromancap\romannumeral #1@}
\headheight 35pt
\rhead{Edward Song \\ \today \\ Math 136}
\title{Math 135}
\author{Edward Song}

\headsep 1.8em
\lfoot{}
\cfoot{}
\rfoot{\small\thepage}

\newtcbtheorem[auto counter,number within=section]{theo}%
  {Theorem}{fonttitle=\bfseries\upshape, fontupper=\slshape,
     arc=0mm, colback=blue!5!white,colframe=blue!75!black}{theorem}

\begin{document}

\section{Vectors in $\rn^n$}


% \begin{tcolorbox}[colback=red!5!white,colframe=red!75!black,title=Recall]
   
% \end{tcolorbox}   


\subsection*{1.8 Cross Product in $\rn^3$}
\begin{tcolorbox}[colback=magenta!5!white,colframe=magenta!75!black,title=Example]
   Given $\overrightarrow{v}, \overrightarrow{u}\in\rn^3$, find $\overrightarrow{w}\in\rn^3$ that is
   perpendicular to both $\overrightarrow{u}$ and $\overrightarrow{v}$.
\end{tcolorbox}   

\begin{tcolorbox}[colback=green!5!white,colframe=green!75!black,title=Definition]
    The \underline{cross product} of $\overrightarrow{u}$ and $\overrightarrow{v}$ is:
    \[\overrightarrow{u}\times\overrightarrow{v} = \begin{bmatrix}
        u^2v_3-u_3v_2 \\
        - (u_1v_3-u_3v_1) \\
        u_1v_2-u_2v_1
    \end{bmatrix} \in \rn^3\]
 \end{tcolorbox}   

 \begin{theo}{Properties}{na}
    \begin{enumerate}
        \item $(\overrightarrow{u}\times\overrightarrow{v})\cdot\overrightarrow{u} = 0$ 
        \item $(\overrightarrow{u}\times\overrightarrow{v})\cdot\overrightarrow{v} - 0$
        \item $||\overrightarrow{u}\times\overrightarrow{v} = ||\overrightarrow{u}||\,||\overrightarrow{v}||\sin\theta$
    \end{enumerate}
 \end{theo}

 \bd{Note:} Cross product satisfies the right-hand rule

 \begin{theo}{Properties}{na}
    \begin{enumerate}
        \item $\overrightarrow{u}\times\overrightarrow{v} = -(\overrightarrow{u}\times\overrightarrow{v})$
        \item $\overrightarrow{u}\times(\overrightarrow{v}+\overrightarrow{w}) = \overrightarrow{u}\times\overrightarrow{w}+\overrightarrow{u}\times\overrightarrow{w}$
        \item $\overrightarrow{u}\times({c\overrightarrow{v}} = c(\overrightarrow{u}\times\overrightarrow{v})) = c\overrightarrow{u} \times \overrightarrow{v}$
    \end{enumerate}
 \end{theo}

\begin{tcolorbox}[colback=magenta!5!white,colframe=magenta!75!black,title=Example]
    \bd{Problem:} $\overrightarrow{e_1} = \begin{bmatrix}
        1\\0\\0
    \end{bmatrix}, \overrightarrow{e^3}=\begin{bmatrix}
        0\\1\\0
    \end{bmatrix}, \overrightarrow{e_3} = \begin{bmatrix}
        0\\0\\1
    \end{bmatrix}$

    Find $\overrightarrow{e_1}\times\overrightarrow{e_2}$
    \begin{enumerate}
        \item Find using right hand rule and coordinate system.
        \item Math 
        $\overrightarrow{e_1}\times\overrightarrow{e_2} = \begin{bmatrix}
            0\\0\\1
        \end{bmatrix} = \overrightarrow{e_3}$
    \end{enumerate}
\end{tcolorbox}   

\begin{tcolorbox}[colback=magenta!5!white,colframe=magenta!75!black,title=Example]
    \bd{Problem:}
    Find $\overrightarrow{e_3}\times\overrightarrow{e_2}$
    
\end{tcolorbox}   


\begin{tcolorbox}[colback=magenta!5!white,colframe=magenta!75!black,title=Example]
    Suppose: $\overrightarrow{v} = \begin{bmatrix}
        2 \\3
    \end{bmatrix}$ and $\overrightarrow{e_1} = \begin{bmatrix}
        1\\0
    \end{bmatrix}$ What is $\text{proj}_{-e_1}\overrightarrow{v}$\\

    \bd{Solution:}
    \begin{align*}
        prof_{-\overrightarrow{e_1}\overrightarrow{v}} &= \frac{\begin{bmatrix}
            2\\3
        \end{bmatrix}\cdot \begin{bmatrix}
            -1\\0
        \end{bmatrix}}{||\begin{bmatrix}
            -1 \\0 
        \end{bmatrix} ||^2} \\
        &=\frac{-2}{1}\begin{bmatrix}
            -1\\0
        \end{bmatrix} \\
        &=\begin{bmatrix}
            2\\0
        \end{bmatrix}
    \end{align*}
\end{tcolorbox}   

\begin{tcolorbox}[colback=green!5!white,colframe=green!75!black,title=Definition]
    Let $\overrightarrow{v}, \overrightarrow{w}\in\rn^n$ with $\overrightarrow{w}\neq 0$. The \bd{projection of} $\overrightarrow{v}$ \bd{onto} 
    $\overrightarrow{w}$ is defined by
    \[\text{perp}_{\overrightarrow{w}} = \overrightarrow{v} - proj_{\overrightarrow{w}}(\overrightarrow{v})\] 
\end{tcolorbox}  

\bd{\underline{Properties}:}
\begin{enumerate}
    \item $proj_{\overrightarrow{w}(\overrightarrow{v})}$ is perpendicular to $perp_{\overrightarrow{w}(\overrightarrow{v})}$
    \item $proj_{\overrightarrow{w}}(c\overrightarrow{v}) = c\cdot proj_{\overrightarrow{w}}(\overrightarrow{v})$
    \item $proj_{\overrightarrow{w}}(\overrightarrow{v} + \overrightarrow{u}) = proj_{\overrightarrow{w}}(\overrightarrow{v}) + proj_{\overrightarrow{w}}(\overrightarrow{v})$
    \item $proj_{\overrightarrow{w}}(proj_{\overrightarrow{w}}(\overrightarrow{v})) = proj_{\overrightarrow{w}}(\overrightarrow{v})$ 
\end{enumerate} 

\bd{Proof of 4:}
\begin{proof}
\begin{align*}
    prof_{\overrightarrow{w}}(proj_{\overrightarrow{w}}(\overrightarrow{v})) &= proj_{\overrightarrow{w}}(\frac{\overrightarrow{v}\cdot\overrightarrow{w}}{||\overrightarrow{w}||^2\overrightarrow{w}}) \\
    &=\frac{\overrightarrow{v}\cdot \overrightarrow{w}}{||\overrightarrow{w}||^2}\cdot \frac{\overrightarrow{w}\cdot\overrightarrow{w}}{||\overrightarrow{w}||^2}\overrightarrow{w} \\
    &=\frac{\overrightarrow{v}\cdot\overrightarrow{w}}{||\overrightarrow{w}||^2}\overrightarrow{w} \\
    &=proj_{\overrightarrow{w}}(\overrightarrow{v})
\end{align*}
\end{proof}

\subsection*{Standard Inner Project in $\mathbb{C}^n$}
Instead of dot product, we define the \underline{\bd{Standard inner product.}}

\begin{tcolorbox}[colback=green!5!white,colframe=green!75!black,title=Definition]
    The \bd{standard inner product} of $\overrightarrow{v} = \begin{bmatrix}
        v_1\\v_2\\\vdots\\v_n
    \end{bmatrix}, \begin{bmatrix}
        w_1\\w_2\\\vdots\\w_n
    \end{bmatrix}\in \mathbb{C}^n$ is \[\langle \overrightarrow{v}, \overrightarrow{w} \rangle = v_1\overrightarrow{w_1} + v_2\overrightarrow{w_2} + \cdots + v_n\overrightarrow{w_n}\]
\end{tcolorbox}  

\begin{tcolorbox}[colback=green!5!white,colframe=green!75!black,title=Definition]
    The \bd{length} of the vector $\overrightarrow{v}\in\mathbb{C}^n$ is $||\overrightarrow{v}|| = \sqrt{\overrightarrow{v}\cdot\overrightarrow{v}}$
\end{tcolorbox} 



\end{document}

