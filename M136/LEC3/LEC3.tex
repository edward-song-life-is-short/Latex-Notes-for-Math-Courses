\documentclass[11pt]{article}

\usepackage{amsmath, amsfonts, amssymb, amsthm}
\usepackage{braket}
\usepackage{fullpage}
\usepackage[top=2cm, bottom=4.5cm, left=2.5cm, right=2.5cm]{geometry}
\usepackage{bbold}
\usepackage{enumitem}
\usepackage{mathtools}

\DeclarePairedDelimiter\ceil{\lceil}{\rceil}
\DeclarePairedDelimiter\floor{\lfloor}{\rfloor}

\usepackage{fancyhdr}
\usepackage{mathrsfs}
\usepackage{xcolor}

\usepackage{listings}
\usepackage{hyperref}
\usepackage[T1]{fontenc}
\usepackage{tabularx}

\usepackage{mathpazo}
\usepackage{xcolor}
\usepackage{float}

\usepackage{graphicx}
\usepackage{subcaption}
\usepackage[export]{adjustbox}
\usepackage{wrapfig}


\usepackage{tikz,lipsum,lmodern}
\usetikzlibrary{calc}
\usetikzlibrary{arrows}
\usepackage{pgfplots}
\usepackage{graphicx}
\makeatletter
\usepackage[most]{tcolorbox}

\setlength{\parindent}{0pt} 
\theoremstyle{plain}
\newtheorem*{theorem}{Theorem}

\theoremstyle{remark}
\newtheorem*{solution}{Solution}

\theoremstyle{plain}
\newtheorem*{claim}{Claim}

\newcolumntype{P}[1]{>{\centering\arraybackslash}p{#1}}
\pagestyle{fancy}
% \newcommand{\ehx}[]{} set new commands
\newcommand{\bd}{\textbf}
\newcommand\course{CSE 3500}

\newcommand{\fa}{\forall}

\newcommand{\nn}{\mathbb{N}}
\newcommand{\z}{\mathbb{Z}}
\newcommand{\rn}{\mathbb{R}}
\newcommand{\stopIndent}{\noindent\underline\bd}


\newcommand*{\rom}[1]{\expandafter\@slowromancap\romannumeral #1@}
\headheight 35pt
\rhead{Edward Song \\ \today \\ Math 136}
\title{Math 135}
\author{Edward Song}

\headsep 1.8em
\lfoot{}
\cfoot{}
\rfoot{\small\thepage}

\newtcbtheorem[auto counter,number within=section]{theo}%
  {Theorem}{fonttitle=\bfseries\upshape, fontupper=\slshape,
     arc=0mm, colback=blue!5!white,colframe=blue!75!black}{theorem}

\begin{document}

\section{Vectors in $\rn^n$}


% \begin{tcolorbox}[colback=red!5!white,colframe=red!75!black,title=Recall]
   
% \end{tcolorbox}   


\subsection*{1.6 Projection, Components, and Perpendicular}
\begin{tcolorbox}[colback=green!5!white,colframe=green!75!black,title=Definition]
    Let $\overrightarrow{v}, \overrightarrow{w}\in\rn^n$ with $\overrightarrow{w}\neq 0$. The \bd{projection of} $\overrightarrow{v}$ \bd{onto} 
    $\overrightarrow{w}$ is defined by
    \[\text{proj}_{\overrightarrow{w}} = (\overrightarrow{v}) - \frac{(\overrightarrow{w}\cdot \overrightarrow{v})}{||\overrightarrow{w}||^2} = \frac{(\overrightarrow{v}\cdot \overrightarrow{w})}{\overrightarrow{w}\cdot\overrightarrow{w}}\overrightarrow{w}\] 
\end{tcolorbox}   

\begin{tcolorbox}[colback=magenta!5!white,colframe=magenta!75!black,title=Example]
    Suppose: $\overrightarrow{v} = \begin{bmatrix}
        2 \\3
    \end{bmatrix}$ and $\overrightarrow{e_1} = \begin{bmatrix}
        1\\0
    \end{bmatrix}$ What is $\text{proj}_{e_1}\overrightarrow{v}$?\\

    \bd{Solution:}
    \begin{align*}
        proj_{\overrightarrow{e^1}}\overrightarrow{v} &=\frac{\begin{bmatrix}
            2\\3
        \end{bmatrix}\cdot\begin{bmatrix}
            1\\0
        \end{bmatrix}}{||\begin{bmatrix}
            1\\0
        \end{bmatrix}||^2} \\
        &= \frac{2}{1}\begin{bmatrix}
            1\\0
        \end{bmatrix} \\
        &= \begin{bmatrix}
            2\\0
        \end{bmatrix}
    \end{align*}
\end{tcolorbox}   

\begin{tcolorbox}[colback=magenta!5!white,colframe=magenta!75!black,title=Example]
    Suppose: $\overrightarrow{v} = \begin{bmatrix}
        2 \\3
    \end{bmatrix}$ and $\overrightarrow{e_1} = \begin{bmatrix}
        1\\0
    \end{bmatrix}$ What is $\text{proj}_{-e_1}\overrightarrow{v}$?\\

    \bd{Solution:}
    \begin{align*}
        prof_{-\overrightarrow{e_1}\overrightarrow{v}} &= \frac{\begin{bmatrix}
            2\\3
        \end{bmatrix}\cdot \begin{bmatrix}
            -1\\0
        \end{bmatrix}}{||\begin{bmatrix}
            -1 \\0 
        \end{bmatrix} ||^2} \\
        &=\frac{-2}{1}\begin{bmatrix}
            -1\\0
        \end{bmatrix} \\
        &=\begin{bmatrix}
            2\\0
        \end{bmatrix}
    \end{align*}
\end{tcolorbox}   

\begin{tcolorbox}[colback=green!5!white,colframe=green!75!black,title=Definition]
    Let $\overrightarrow{v}, \overrightarrow{w}\in\rn^n$ with $\overrightarrow{w}\neq 0$. The \bd{projection of} $\overrightarrow{v}$ \bd{onto} 
    $\overrightarrow{w}$ is defined by
    \[\text{perp}_{\overrightarrow{w}} = \overrightarrow{v} - proj_{\overrightarrow{w}}(\overrightarrow{v})\] 
\end{tcolorbox}  

\bd{\underline{Properties}:}
\begin{enumerate}
    \item $proj_{\overrightarrow{w}(\overrightarrow{v})}$ is perpendicular to $perp_{\overrightarrow{w}(\overrightarrow{v})}$
    \item $proj_{\overrightarrow{w}}(c\overrightarrow{v}) = c\cdot proj_{\overrightarrow{w}}(\overrightarrow{v})$
    \item $proj_{\overrightarrow{w}}(\overrightarrow{v} + \overrightarrow{u}) = proj_{\overrightarrow{w}}(\overrightarrow{v}) + proj_{\overrightarrow{w}}(\overrightarrow{v})$
    \item $proj_{\overrightarrow{w}}(proj_{\overrightarrow{w}}(\overrightarrow{v})) = proj_{\overrightarrow{w}}(\overrightarrow{v})$ 
\end{enumerate} 

\bd{Proof of 4:}
\begin{proof}
\begin{align*}
    prof_{\overrightarrow{w}}(proj_{\overrightarrow{w}}(\overrightarrow{v})) &= proj_{\overrightarrow{w}}(\frac{\overrightarrow{v}\cdot\overrightarrow{w}}{||\overrightarrow{w}||^2\overrightarrow{w}}) \\
    &=\frac{\overrightarrow{v}\cdot \overrightarrow{w}}{||\overrightarrow{w}||^2}\cdot \frac{\overrightarrow{w}\cdot\overrightarrow{w}}{||\overrightarrow{w}||^2}\overrightarrow{w} \\
    &=\frac{\overrightarrow{v}\cdot\overrightarrow{w}}{||\overrightarrow{w}||^2}\overrightarrow{w} \\
    &=proj_{\overrightarrow{w}}(\overrightarrow{v})
\end{align*}
\end{proof}

\subsection*{Standard Inner Project in $\mathbb{C}^n$}
Instead of dot product, we define the \underline{\bd{Standard inner product.}}

\begin{tcolorbox}[colback=green!5!white,colframe=green!75!black,title=Definition]
    The \bd{standard inner product} of $\overrightarrow{v} = \begin{bmatrix}
        v_1\\v_2\\\vdots\\v_n
    \end{bmatrix}, \begin{bmatrix}
        w_1\\w_2\\\vdots\\w_n
    \end{bmatrix}\in \mathbb{C}^n$ is \[\langle \overrightarrow{v}, \overrightarrow{w} \rangle = v_1\overrightarrow{w_1} + v_2\overrightarrow{w_2} + \cdots + v_n\overrightarrow{w_n}\]
\end{tcolorbox}  

\begin{tcolorbox}[colback=green!5!white,colframe=green!75!black,title=Definition]
    The \bd{length} of the vector $\overrightarrow{v}\in\mathbb{C}^n$ is $||\overrightarrow{v}|| = \sqrt{\overrightarrow{v}\cdot\overrightarrow{v}}$
\end{tcolorbox} 

\begin{theo}{Property 1.5.3}{name}
    \begin{enumerate}
        \item $\overrightarrow{u},\overrightarrow{v}, \overrightarrow{w} \in \rn$
        \item $\overrightarrow{u}\cdot\overrightarrow{v} = \overrightarrow{v}\cdot\overrightarrow{u}$
        \item $(\overrightarrow{u}+\overrightarrow{v})\cdot\overrightarrow{w} = \overrightarrow{u}\overrightarrow{w}+\overrightarrow{v}\overrightarrow{w}$
        \item $(\overrightarrow{u}\cdot\overrightarrow{v})\cdot\overrightarrow{w} = \overrightarrow{v}\cdot(\overrightarrow{u}\cdot\overrightarrow{w})$
        \item $\overrightarrow{v}\cdot \overrightarrow{v}\geq 0$
    \end{enumerate}
\end{theo}

\subsection*{Geometry in $\rn^2$}
\begin{tcolorbox}[colback=green!5!white,colframe=green!75!black,title=Definition]
    The \bd{length} of the vector $\overrightarrow{v}\in \rn^n$ is $||\overrightarrow{v}|| = \sqrt{\overrightarrow{v}\cdot\overrightarrow{v}}$\\
    
    \bd{Aside:} In $\rn^1:$ \[||\overrightarrow{v}||=||[v_1]|| = \sqrt{v_1^2} = |v|\]
\end{tcolorbox}   
\begin{theo}{Properties of Length}{name}
    \begin{enumerate}
        \item $||\overrightarrow{0}|| = 0$
        \item $||c \cdot \overrightarrow{v}|| = |c|\cdot||\overrightarrow{v}||$
        \item $||\overrightarrow{v} + \overrightarrow{u}|| \neq ||\overrightarrow{v}|| + ||\overrightarrow{u}||$
        \item $||\overrightarrow{v} + \overrightarrow{u}||\leq ||\overrightarrow{v}|| + ||\overrightarrow{u}||$
    \end{enumerate}
\end{theo}

\bd{Importance of dot product:} It gives angles between vectors in $\rn^2$!

\begin{tcolorbox}[colback=green!5!white,colframe=green!75!black,title=Definition]
    $\overrightarrow{v} \in \rn^n$ is a \bd{unit vector} if $||\overrightarrow{v} = 1||$
\end{tcolorbox} 

\begin{tcolorbox}[colback=green!5!white,colframe=green!75!black,title=Definition]
    When $\overrightarrow{v}\in \rn^n$ is a non-zero vector, we can produce a unit vector
    \[\hat{v} = \frac{\overrightarrow{v}}{||\overrightarrow{v}||}\]

    in the direction of $\overrightarrow{v}$ by scaling $\overrightarrow{v}$. This process is called normalization.
\end{tcolorbox} 

\begin{tcolorbox}[colback=green!5!white,colframe=green!75!black,title=Definition]
    Let $\overrightarrow{v}$ and $\overrightarrow{u}$ be non-zero vectors in $\rn^n$. The angle $\theta$, in radians $(0\leq \theta\pi)$, between $\overrightarrow{u}$
    and $\overrightarrow{v}$ is such that \[\overrightarrow{v}\cdot\overrightarrow{u} = ||\overrightarrow{u}||\cdot||\overrightarrow{v}||\cos\theta, \,\text{that is}\,\theta = \arccos(\frac{\overrightarrow{u}\cdot\overrightarrow{v}}{||\overrightarrow{v}||\cdot||\overrightarrow{u}||})\]
\end{tcolorbox} 


\begin{tcolorbox}[colback=magenta!5!white,colframe = magenta!75!black,title=Example]
    \bd{Problem:} Given 2 vectors, $\begin{bmatrix}
        0\\1
    \end{bmatrix}$ and $\begin{bmatrix}
        1\\0
    \end{bmatrix}$ find $\theta$.

    \bd{Solution:} \\

   \begin{align*}
       \cos\theta &= \frac{\begin{bmatrix}
           0\\1
       \end{bmatrix}\cdot \begin{bmatrix}
           1\\0
       \end{bmatrix}}{||\begin{bmatrix}
           0\\1
       \end{bmatrix}||\cdot||\begin{bmatrix}
           1\\0
       \end{bmatrix}||}\\
       &=\frac{0}{1} = 0\\
       \theta&=\frac{\pi}{2}
   \end{align*}
\end{tcolorbox} 

\begin{tcolorbox}[colback=green!5!white,colframe=green!75!black,title=Definition]
    Let $\overrightarrow{u}, \overrightarrow{v}\in \rn^n$. We say $\overrightarrow{u}$ and $\overrightarrow{v}$ are 
    \underline{perpendicular} (or \underline{orthogonal}) if $\overrightarrow{u}\cdot \overrightarrow{v} = 0$
\end{tcolorbox} 
\begin{tcolorbox}[colback=magenta!5!white,colframe = magenta!75!black,title=Example]
    \bd{Problem:} Find a non-zero vector in $\rn^2$ that is orthogonal to $\begin{bmatrix}
        1\\2
    \end{bmatrix}$

    \bd{Solution:}
    $\begin{bmatrix}
        2\\-1
    \end{bmatrix}, \begin{bmatrix}
        -2\\1
    \end{bmatrix}, \begin{bmatrix}
        6\\-3
    \end{bmatrix}$ 

\end{tcolorbox} 

\begin{tcolorbox}[colback=magenta!5!white,colframe = magenta!75!black,title=Example]
    \bd{Problem:} Find a non-zero vector perpendicular to $\begin{bmatrix}
        a\\b
    \end{bmatrix}$

    \bd{Solution:} \\
    \begin{align*}
        &\begin{bmatrix}
            -b\\a
        \end{bmatrix}
        \cdot \begin{bmatrix}
            a\\b
        \end{bmatrix} \\
        =& -ba+ab \\=&0
    \end{align*}

    Therefore, $\begin{bmatrix}
        -b \\a
    \end{bmatrix}$ is a solution.
\end{tcolorbox} 

\end{document}

