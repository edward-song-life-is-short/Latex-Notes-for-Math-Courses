\documentclass[11pt]{article}

\usepackage{amsmath, amsfonts, amssymb, amsthm}
\usepackage{braket}
\usepackage{fullpage}
\usepackage[top=2cm, bottom=4.5cm, left=2.5cm, right=2.5cm]{geometry}
\usepackage{bbold}
\usepackage{enumitem}
\usepackage{mathtools}

\DeclarePairedDelimiter\ceil{\lceil}{\rceil}
\DeclarePairedDelimiter\floor{\lfloor}{\rfloor}

\usepackage{fancyhdr}
\usepackage{mathrsfs}
\usepackage{xcolor}

\usepackage{listings}
\usepackage{hyperref}
\usepackage[T1]{fontenc}
\usepackage{tabularx}

\usepackage{mathpazo}
\usepackage{xcolor}
\usepackage{float}

\usepackage{graphicx}
\usepackage{subcaption}
\usepackage[export]{adjustbox}
\usepackage{wrapfig}


\usepackage{tikz,lipsum,lmodern}
\usetikzlibrary{calc}
\usetikzlibrary{arrows}
\usepackage{pgfplots}
\usepackage{graphicx}
\makeatletter
\usepackage[most]{tcolorbox}

\setlength{\parindent}{0pt} 
\theoremstyle{plain}
\newtheorem*{theorem}{Theorem}

\theoremstyle{remark}
\newtheorem*{solution}{Solution}

\theoremstyle{plain}
\newtheorem*{claim}{Claim}

\newcolumntype{P}[1]{>{\centering\arraybackslash}p{#1}}
\pagestyle{fancy}
% \newcommand{\ehx}[]{} set new commands
\newcommand{\bd}{\textbf}
\newcommand\course{CSE 3500}

\newcommand{\fa}{\forall}

\newcommand{\nn}{\mathbb{N}}
\newcommand{\z}{\mathbb{Z}}
\newcommand{\rn}{\mathbb{R}}
\newcommand{\stopIndent}{\noindent\underline\bd}


\newcommand*{\rom}[1]{\expandafter\@slowromancap\romannumeral #1@}
\headheight 35pt
\rhead{Edward Song \\ \today \\ Math 136}
\title{Math 136}
\author{Edward Song}

\headsep 1.8em
\lfoot{}
\cfoot{}
\rfoot{\small\thepage}

\newtcbtheorem[auto counter,number within=section]{theo}%
  {Theorem}{fonttitle=\bfseries\upshape, fontupper=\slshape,
     arc=0mm, colback=blue!5!white,colframe=blue!75!black}{theorem}

\begin{document}

\section*{2 Spans, Sections, Lines}


% \begin{tcolorbox}[colback=red!5!white,colframe=red!75!black,title=Recall]
   
% \end{tcolorbox}   

\subsection*{2.1 Linear Combinations and Span}

\begin{tcolorbox}[colback=red!5!white,colframe=red!75!black,title=Definition: Recall]
   Recall definition: Let $\overrightarrow{v_1}, \overrightarrow{v_2}, \cdots, \overrightarrow{v_n}\in \mathbb{F}^n$. A \underline{linear combination} of 
   $\overrightarrow{v_1}\cdots, \overrightarrow{v_n}$ is a vector of the form:
   \[c_1\overrightarrow{v_1}+c_2\overrightarrow{v_2}+\cdots+c_n\overrightarrow{v_n}\]
\end{tcolorbox}   

\begin{tcolorbox}[colback=magenta!5!white,colframe=magenta!75!black,title=Problem 1 ]
    \begin{enumerate}
        \item In $\rn^2$: \[2\begin{bmatrix}
            1\\1
        \end{bmatrix}+(-1)\begin{bmatrix}
            1\\2
        \end{bmatrix}=\begin{bmatrix}
            1\\0
        \end{bmatrix}\] is a linear combination of $\begin{bmatrix}
            1\\1
        \end{bmatrix}$ and $\begin{bmatrix}
            1\\2
        \end{bmatrix}$
        \item In $\mathbb{F}^n$\\
         $\overrightarrow{v} = 1\cdot \overrightarrow{v}$ \\
         $\overrightarrow{v}=1\cdot\overrightarrow{v}+0\overrightarrow{u_1}+0\overrightarrow{u_2}+\cdots+0\overrightarrow{u_n}$
        \item In $\mathbb{F}^n \overrightarrow{0} = 0\overrightarrow{v_1}+\cdots+0\overrightarrow{v_n}$
        
    \end{enumerate}
\end{tcolorbox}   

\begin{tcolorbox}[colback=green!5!white,colframe=green!75!black,title=Problem 2 ]
    Let $\overrightarrow{v_1}, \cdots, \overrightarrow{v_n}\in\mathbb{F}^n$. The \underline{span} of $\overrightarrow{v_1}, \cdots, \overrightarrow{v_n}\in \mathbb{F}^n$
    is the set span $\{ \overrightarrow{v_1}, \cdots, \overrightarrow{v_n}\} = \{ c_1\overrightarrow{v_1}+c_2\overrightarrow{v_2}+\cdots+c_n\overrightarrow{v_n}: c_1, \cdots, c_n \in \mathbb{F}\}$
\end{tcolorbox}   

\bd{Warning:}\\

\[span\{\overrightarrow{v_1}, \cdots, \overrightarrow{v_n}\} = c_1\overrightarrow{v_1}+\cdots + c_n\overrightarrow{v_n}\]
The left side is a set, while the right side is a vector.

\begin{tcolorbox}[colback=magenta!5!white,colframe=magenta!75!black,title=Problem 4]
   \begin{enumerate}
       \item T/F: $\rn^2=span\{\overrightarrow{e_1}, \overrightarrow{e_2}\}=span \{\begin{bmatrix}
           1\\0
       \end{bmatrix}\}, \begin{bmatrix}
           1\\0
       \end{bmatrix}\}$

       \begin{proof}
           \begin{align*}
               &span\{\begin{bmatrix}
                   1\\0
               \end{bmatrix}, \begin{bmatrix}
                   0\\1
               \end{bmatrix}\}\\
               =&\{a\begin{bmatrix}
                   1\\0
               \end{bmatrix}+b\begin{bmatrix}
                   0\\1
               \end{bmatrix}:a,b\in\rn\}\\
               =&\{\begin{bmatrix}
                   a\\b
               \end{bmatrix}:a,b\in\rn\}\\
               =&\rn^2
           \end{align*}
       \end{proof}

       \item T/F: $\rn^2=span\{\begin{bmatrix}
           1\\1
       \end{bmatrix}, \begin{bmatrix}
           3\\4
       \end{bmatrix}\}$
   \end{enumerate}
\end{tcolorbox}   

\begin{tcolorbox}[colback=magenta!5!white,colframe=magenta!75!black,title=Problem 5]
    Evaluate $\int_0^2|x^2-3x+2|dx$\\

    \begin{align*}
        \int_0^2|x^2-3x+2|dx &=\int_0^1(x^2-3x+2)dx+\int_1^2(-x^2+3x-2)dx \\
        &=(\frac{1}{3}x^3-\frac{3}{2}x^2+2x)\Big|_0^1+(-\frac{1}{3}x^3+\frac{3}{2}x^2-2x)\Big|_1^2 \\
        &=(\frac{1}{3}-\frac{3}{2}+2)-0+(-\frac{8}{3}+6-4)-(-\frac{1}{3}+\frac{3}{2}-2) \\
        &= 1
    \end{align*}
\end{tcolorbox}   



\end{document}