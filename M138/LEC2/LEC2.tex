\documentclass[11pt]{article}

\usepackage{amsmath, amsfonts, amssymb, amsthm}
\usepackage{braket}
\usepackage{fullpage}
\usepackage[top=2cm, bottom=4.5cm, left=2.5cm, right=2.5cm]{geometry}
\usepackage{bbold}
\usepackage{enumitem}
\usepackage{mathtools}
\DeclarePairedDelimiter\ceil{\lceil}{\rceil}
\DeclarePairedDelimiter\floor{\lfloor}{\rfloor}

\usepackage{fancyhdr}
\usepackage{mathrsfs}
\usepackage{xcolor}
\usepackage{graphicx}
\usepackage{listings}
\usepackage{hyperref}
\usepackage[T1]{fontenc}
\usepackage{tabularx}

\usepackage{mathpazo}
\usepackage{xcolor}

\usepackage{tikz,lipsum,lmodern}
\usetikzlibrary{calc}
\usetikzlibrary{arrows}
\usepackage{pgfplots}


\usepackage[most]{tcolorbox}

\setlength{\parindent}{0pt} 
\theoremstyle{plain}
\newtheorem*{theorem}{Theorem}

\theoremstyle{remark}
\newtheorem*{solution}{Solution}

\theoremstyle{plain}
\newtheorem*{claim}{Claim}

\newcolumntype{P}[1]{>{\centering\arraybackslash}p{#1}}
\pagestyle{fancy}
% \newcommand{\ehx}[]{} set new commands
\newcommand{\bd}{\textbf}
\newcommand\course{CSE 3500}

\newcommand{\fa}{\forall}

\newcommand{\nn}{\mathbb{N}}
\newcommand{\z}{\mathbb{Z}}
\newcommand{\rn}{\mathbb{R}}
\newcommand{\stopIndent}{\noindent\underline\bd}

\headheight 35pt

\rhead{Edward Song \\ \today \\ Math 138}
\title{Math 135}
\author{Edward Song}

\headsep 1.8em
\lfoot{}
\cfoot{}
\rfoot{\small\thepage}

\newtcbtheorem[auto counter,number within=section]{theo}%
  {Theorem}{fonttitle=\bfseries\upshape,
     arc=0mm, colback=blue!5!white,colframe=blue!75!black}{theorem}

\begin{document}

\section {Integration}

Definitions that our prof briefly mentioned but did not go over carefully:
\begin{tcolorbox}[colback=green!5!white,colframe=green!75!black,title=Definition: Right-hand Riemann Sum]
    The \emph{right-hand Riemann Sum} for $f$ with respect to the partition \emph{P} is the Riemann sum 
    \emph{R} obtained from \emph{P} by choosing $c_i$ to be $t_i$, the right-hand endpoint of $[t_{i-1}, t_i]$. 
    That is \[R = \sum_{i=1}^{n} f(t_i)\Delta t_i\]

    If $P^{(n)}$ is the regular n-partition, we denote the right-hand Riemann sum by:
    \begin{align*}
        R_n = \sum_{i=1}^{n}f(t_i)\Delta t_i &= \sum_{i=1}^{n}f(t_i)\frac{b-a}{n} \\
        &=\sum_{i=1}^{n}f(a+i(\frac{b-a}{n}))(\frac{b-a}{n})
    \end{align*}
 \end{tcolorbox} 
 
 \begin{tcolorbox}[colback=green!5!white,colframe=green!75!black,title=Definition: Left-hand Riemann Sum]
    The \emph{right-hand Riemann Sum} for $f$ with respect to the partition \emph{P} is the Riemann sum 
    \emph{R} obtained from \emph{P} by choosing $c_i$ to be $t_i$, the right-hand endpoint of $[t_{i-1}, t_i]$. 
    That is \[L = \sum_{i=1}^{n} f(t_{i-1})\Delta t_i\]

    If $P^{(n)}$ is the regular n-partition, we denote the right-hand Riemann sum by:
    \begin{align*}
        R_n = \sum_{i=1}^{n}f(t_{i-1})\Delta t_i &= \sum_{i=1}^{n}f(t_{i-1})\frac{b-a}{n} \\
        &=\sum_{i=1}{n}f(a+(i-1)(\frac{b-a}{n}))(\frac{b-a}{n})
    \end{align*}
 \end{tcolorbox} 



\subsection*{1.2 Riemann Sums and the Definite Integral}

\begin{tcolorbox}[colback=green!5!white,colframe=green!75!black,title=Definition]
   The definite integral of a function $f$ from $a$ to $b$ is \[ \int_{a}^{b} f(x) \,dx = \lim_{n\to \infty}
   \sum_{i=1}^{n} f(x_i)\Delta x; \quad \Delta x = \frac{b-a}{n}, x_i \in [x_{i-1},x_i] \]

    If this limit exists, then $f$ is said to be integrable on the inteveral $[a,b]$
\end{tcolorbox}   

Compare to textbook definition:
\begin{tcolorbox}[colback=green!5!white,colframe=green!75!black,title=Definition]
    We say that a bounded function $f$ is \emph{integrable} on $[a,b]$ if there exists a unique number 
    $I\in\rn$ such that if whenever $\{P_n\}$ is a sequence of partitions with $\lim_{n\to\infty} ||P_n||=0$ and 
    $\{S_n\}$ is any sequence of Riemann sums associated with the $P_n$'s, we have \[\lim_{n\to\infty}S_n = I\]
    
    In this case, we call the $I$ the integral of $f$ over $[a,b]$ and denote it by:
    \[\int_{a}^{b} f(t)\,dt\]
 \end{tcolorbox}   


 $\linebreak$
 \bd{Important notes:}
 \begin{itemize}
     \item the width $\Delta x$ is not the same
     \item the sum $\sum_{i=1}^{n} f(x_i) \Delta x$ is called a Riemann sum
     \item the symbol $\int$ is called an integral sign 
     \item the definite integral represnts a number, does not depend on x, called a \underline{dummy variable}
     since it can be replaced by another variable 
     \item if $f(x)\geq 0$ on $[a,b]$, then $\int_{a}^{b}f(x)\,dx$ can be interprested as the area $A$
 \end{itemize}

 \begin{theo}{Why does our prof not mention this?}{summation}
    Let $f$ be continuous on $[a,b]$. Then $f$ is integrable on $[a,b]$. Moreover
    \[\int_{a}^{b} f(t)\,dt = \lim_{n\to\infty}S_n\]
    where 
    \[S_n = \sum_{i=1}^{n}f(c_i)\Delta t_i\]

    is any Riemann summ associated with the regular n-partitions. In particular:
    \[\int_{a}^{b} f(t)\,dt = \lim^{n\to\infty}R_n = \lim_n\to\infty \sum_{i=1}^n f(t_i)\frac{b-a}{n}\]

    and 
    \[\int_{a}^{b} f(t)\,dt = \lim^{n\to\infty}L_n = \lim_n\to\infty \sum_{i=1}^n f(t_{i-1})\frac{b-a}{n}\]
  \end{theo}

 \begin{theo}{No name}{summation}
    If $f$ is continuous on $[a,b]$, then $f$ is integrable, that is, the definite integral $\int_{a}^{b}f(x)\,dx$ exists.
  \end{theo}

$\linebreak$
\bd{\underline{Note:}} If $f$ is discontinuous, then $\int_{a}^{b}f(x)\,dx$ may exist. For example if $f$ has a jump discontinuity as shown,
then $\int_a^bf(x)\,dx$ exists.

\subsection*{1.3 Properties of the Definite Integral}
\begin{theo}{Properties of Integrals}{sum}
    Assume that $f$ and $g$ are integrable on the interval $[a,b]$. Then:
    \begin{enumerate}
        \item For any $c\in \rn$, $\int_a^b c\cdot f(t)\,dt = c\int_a^b\,f(t)\,dt$
        \item $\int_a^b(f+g)(t)\,dt = \int_a^bf(t)\,dt + \int_a^b\,f(t)\,dt$
        \item $\int_a^b\,f(x)\,dx = \int_a^c\,f(x)\,dx +\int_c^b\,f(x)\,dx $
        \item $\int_a^a\,f(x)\,dx= 0$
        \item $\int_a^b\,f(x)\,dx = -\int_b^a\,f(x)\,dx$
        \item If $f(x)\geq 0$ for $a \leq x \leq b$, then $\int_a^b\,f(x)\,dx \geq 0$
        \item If $f(x)\geq g(x)$ for $a \leq x \leq b$, then $\int_a^b\,f(x)\,dx \geq g(x)$
        \item If $m\leq f(x)\leq M$ for $a \leq x \leq b$, then $m(b-a) \leq \int_a^b\,f(x)\,dx\leq M(b-a)$ \\(Integral Squeeze Theorem)
        \item $|\int_a^b\,f(x)\,dx| \leq \int_a^b\,|f(x)|\,dx $ (Integral Triangle Inequality)
        \item If $f(x)$ is an odd function, that is, $f(-x) = -f(x)$, then \[\int_{-a}^a\,f(x)\,dx\]
        \item If $f(x)$ is an even function, that is, $f(-x) = f(x)$, then \[\int_{-a}^{a}\,f(x)\,dx = 2\int_{0}^a\,f(x)\,dx\]
    \end{enumerate}
\end{theo}

\subsection*{1.4 Average Value of a Function}
\begin{tcolorbox}[colback=red!5!white,colframe=red!75!black,title=Recall]
    Recall that the average value of $n$ real numbers, $\alpha_1, \alpha_2, \cdots, \alpha_n$ is given by:
    \[\frac{\alpha_1, \alpha_2, \cdots, \alpha_n}{n} = \frac{\sum_{i=1}^{n}\alpha_i}{n}\]
\end{tcolorbox}   

If we denote the average value of $f(x)$ by $\tilde{f}$, then we expect that \[\tilde{f} \approx \frac{\sum_{i=1}^nf(x_i)(\frac{b-a}{n})}{b-a}\]

We can prove this. 

\begin{proof}
    Suppose $f(x)$ has a min value of $m$ and a max value of $M$ on $[a,b]$. Then we expect that $m\leq \tilde{f}\leq M$. We can show this as follows:

    \begin{align*}
        m \leq &f(x) \leq M, a \leq x \leq b \\
        m(b-a)\leq &\int_a^b\,f(x)\,dx\leq M(b-a) \quad \text{1.3}\\
        m\leq &\frac{1}{b-a}\int_a^b\,f(x)\,dx\leq M \quad \text{Integral MVT}
    \end{align*}

    Since $f(x)$ is continuous, then by IVT $\exists c \in [a,b]$ such that \[\tilde{f} = f(c) = \frac{1}{b-a}\int_a^b\,f(x)\,dx\] or \[ f(c)(b-a) = \int_a^b\,f(x)\,dx\]
\end{proof}

\end{document}

