\documentclass[11pt]{article}

\usepackage{amsmath, amsfonts, amssymb, amsthm}
\usepackage{braket}
\usepackage{fullpage}
\usepackage[top=2cm, bottom=4.5cm, left=2.5cm, right=2.5cm]{geometry}
\usepackage{bbold}
\usepackage{enumitem}
\usepackage{mathtools}

\DeclarePairedDelimiter\ceil{\lceil}{\rceil}
\DeclarePairedDelimiter\floor{\lfloor}{\rfloor}

\usepackage{fancyhdr}
\usepackage{mathrsfs}
\usepackage{xcolor}

\usepackage{listings}
\usepackage{hyperref}
\usepackage[T1]{fontenc}
\usepackage{tabularx}

\usepackage{mathpazo}
\usepackage{xcolor}
\usepackage{float}

\usepackage{graphicx}
\usepackage{subcaption}
\usepackage[export]{adjustbox}
\usepackage{wrapfig}


\usepackage{tikz,lipsum,lmodern}
\usetikzlibrary{calc}
\usetikzlibrary{arrows}
\usepackage{pgfplots}
\usepackage{graphicx}
\makeatletter
\usepackage[most]{tcolorbox}

\setlength{\parindent}{0pt} 
\theoremstyle{plain}
\newtheorem*{theorem}{Theorem}

\theoremstyle{remark}
\newtheorem*{solution}{Solution}

\theoremstyle{plain}
\newtheorem*{claim}{Claim}

\newcolumntype{P}[1]{>{\centering\arraybackslash}p{#1}}
\pagestyle{fancy}
% \newcommand{\ehx}[]{} set new commands
\newcommand{\bd}{\textbf}
\newcommand\course{CSE 3500}

\newcommand{\fa}{\forall}

\newcommand{\nn}{\mathbb{N}}
\newcommand{\z}{\mathbb{Z}}
\newcommand{\rn}{\mathbb{R}}
\newcommand{\stopIndent}{\noindent\underline\bd}


\newcommand*{\rom}[1]{\expandafter\@slowromancap\romannumeral #1@}
\headheight 35pt
\rhead{Edward Song \\ \today \\ Math 138}
\title{Math 135}
\author{Edward Song}

\headsep 1.8em
\lfoot{}
\cfoot{}
\rfoot{\small\thepage}

\newtcbtheorem[auto counter,number within=section]{theo}%
  {Theorem}{fonttitle=\bfseries\upshape, fontupper=\slshape,
     arc=0mm, colback=blue!5!white,colframe=blue!75!black}{theorem}

\begin{document}

\section*{2 Techniques of Integration}


% \begin{tcolorbox}[colback=red!5!white,colframe=red!75!black,title=Recall]
   
% \end{tcolorbox}   

\subsection*{2.1 Inverse Trigonometric Substitutions}

This is a special case of the method of substitution involving trig functions. Integrals involving radicals:
\[\sqrt{a^2-b^2x^2},\sqrt{a^2+b^2x^2},\sqrt{b^2x^2-a^2}\] can be evaluated by making the 
following substitutions:

\begin{align*}
    \begin{array}{|c|c|c|} \hline
        \text{Radical} & \text{Trig Subsitution} &\text{Relevant Identity}  \\\hline
        \sqrt{a^2-b^2x^2} & bx=a\sin\theta &1-\sin^2\theta=\cos^2\theta \\\hline
        \sqrt{\sqrt{a^2+b^2x^2}} & bx=a\tan\theta &1+\tan^2\theta=\sec^2\theta \\\hline
        \sqrt{b^2x^2-a^2} & bx=a\sec\theta &\sec^2\theta-1=\tan^2\theta \\\hline
    \end{array}
\end{align*}

This reduces the integral to a trigonometric integral which may be easier to evaluate. When using this method
the following integrals arise often and hence should be added to our list of elementary integrals:
\begin{align*}
    \int\tan\theta d\theta &= \ln |\sec\theta |+C\\
    \int\cot\theta d\theta &=\ln |\csc\theta | + C\\
    \int\sec\theta d\theta &= \ln |\sec\theta+\tan\theta |+C\\
    \int\csc\theta d\theta &= -\ln |\csc\theta +\cot\theta|+C
\end{align*}

The first two can be shown through substitution and the last two can be verified through differentiation.



\begin{tcolorbox}[colback=magenta!5!white,colframe=magenta!75!black,title=Example]
   \begin{enumerate}
       \item Evaluate $\int\frac{dx}{4+x^2}$. \\
       
       Let $x=2\tan\theta$, then $dx=2\sec^2\theta d\theta$. Thus,
       \[\int\frac{dx}{4+x^2}=\int\frac{2\sec^2\theta d\theta}{4+4\tan^2\theta}=
       \int\frac{sec^2\theta d\theta}{\sqrt{1+\tan^2\theta}}=\int\frac{\sec^2\theta d\theta}{|\sec\theta|}\]
       Now, \[\theta=\tan^{-1}\frac{x}{2}\rightarrow \theta\in(-\frac{\pi}{2},\frac{\pi}{2})\rightarrow \sec\theta>0\rightarrow |\sec\theta|=\sec\theta\]
       \[\int\frac{dx}{4+4x^2}=\int\sec\theta d\theta = \ln|\sec\theta+\tan\theta|+C\]
       We need to express this in terms of x. To do this, draw a triangle. We see that \[\sec\theta = \frac{1}{\cos\theta}=\frac{\sqrt{4+x^2}}{2}\]
       \begin{align*}
           \int\frac{dx}{\sqrt{4+x^2}}&=\ln |\frac{\sqrt{4+x^2}}{2}+\frac{x}{2}|+C\\
           &=\ln |x+\sqrt{4+x^2}|+K
       \end{align*}

       Recall that $\ln(\frac{A}{B})=\ln A-\ln B$, so $K=C-\ln 2$.
       \item Evaluate \[\int\frac{dx}{x^2\sqrt{x^2-16}}=\int\frac{4\sec\theta\tan\theta\,d\theta}{16\sec^2\theta\sqrt{16(\sec^2\theta-1)}}=\frac{1}{16}\int\frac{\tan\theta\,d\theta}{\sec\theta|\tan\theta|}\]
       Now $\theta=\sec^{-1}\frac{x}{4}$ if we restrict $\theta\in[0,\frac{\pi}{2}]\cup[\pi,\frac{3\pi}{2}]$, then $\tan\theta>0\rightarrow |\tan\theta|=\tan\theta$.
       \[\int\frac{dx}{x^2\sqrt{x^2-16}}=\frac{1}{16}\int\frac{d\theta}{\sec\theta}=\frac{1}{16}\cos\theta d\theta =\frac{1}{16}\sin\theta +C=\frac{\sqrt{x^2-16}}{16x}+C\]

       But from the triangle $\sin\theta=\frac{\sqrt{x^2-16}}{x}$. \\

       \underline{\bd{Note:}} If we choose $\theta\in[0,\frac{\pi}{2})\cup (\frac{\pi}{2},\pi]$, then we would need 
       to consider two cases: $\tan\theta>0,\tan\theta<0$.
   \end{enumerate}
\end{tcolorbox}   


\begin{tcolorbox}[colback=magenta!5!white,colframe=magenta!75!black,title=Example ]
    \begin{enumerate} 
        \item Evaluate $\int\frac{dx}{(5-4x-x^2)^{\frac{5}{2}}}$:
        
        We first complete the square: $5-4x-x^2=9-(x+2)^2$. Let $x+2=3\sin\theta\rightarrow dx=3\cos\theta d\theta$. Then:
        \begin{align*}
            \int\frac{dx}{5-4x-x^2}^{\frac{5}{2}}&=\int\frac{dx}{[9-(x+2)^2]^{\frac{5}{2}}}\\
            &=\frac{1}{3^4}\int\frac{d\theta}{\cos^2\theta}=\frac{1}{81}\int\sec^4\theta d\theta\\
            &=\frac{1}{81}\int\sec^2\theta(1+\tan^2\theta)d\theta \\
            &=\frac{1}{81}\int\sec^2\theta d\theta +\frac{1}{81}\int\sec^2\theta \tan^2\theta d\theta\\
            &=\frac{1}{81}\tan\theta+\frac{1}{81}[\frac{1}{3}\tan^3\theta]+C\\
            &=\frac{1}{81}[\frac{x+2}{9-(x+2)^2}]+\frac{1}{343}\Big[\frac{(x+2^3)}{[9-(x+2)^2]}^{\frac{3}{2}}\Big]+C
        \end{align*}

        \item Evaluate the definite integral $\int_0^{\frac{3\sqrt{3}}{2}}\frac{x^3dx}{(4x^2+9)^{\frac{3}{2}}}$\\
        
        Let $2x=3\tan\theta$, then $2dx=4\sec^2\theta\,d\theta$. Thus,
        \begin{align*}
            \int_0^{\frac{3\sqrt{3}}{2}}\frac{x^3dx}{(4x^2+9)^{\frac{3}{2}}}&=\frac{3}{16}\int_0^{\frac{\pi}{3}}\frac{\tan^3\theta\sec^2\theta}{\sec^3\theta}d\theta\\
            &=\frac{3}{16}\int_0^{\frac{\pi}{3}}\frac{\tan^3\theta}{\sec\theta}d\theta =\frac{3}{16}\int_0^{\frac{\pi}{3}}\frac{\sin^3\theta}{\cos^2\theta}d\theta\\
            &=\frac{3}{16}\int_0^{\frac{\pi}{3}}\frac{\sin\theta(1-\cos^2\theta)}{\cos^2\theta}d\theta 
        \end{align*}

        Now make the substitution $u=\cos\theta, du=-\sin\theta\,d\theta$.
        \begin{align*}
            &=-\frac{3}{16}\int_1^{\frac{1}{2}}]\frac{(1-u)^2}{u^2}du=\frac{3}{16}\int_{\frac{1}{2}}^1[\frac{1}{u^2}-1]du\\
            &=\frac{3}{16}[-\frac{1}{u}-u]\Big|_{\frac{1}{2}}^1=\frac{3}{32}
        \end{align*}
    \end{enumerate}
\end{tcolorbox}   




\end{document}