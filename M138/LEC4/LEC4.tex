\documentclass[11pt]{article}

\usepackage{amsmath, amsfonts, amssymb, amsthm}
\usepackage{braket}
\usepackage{fullpage}
\usepackage[top=2cm, bottom=4.5cm, left=2.5cm, right=2.5cm]{geometry}
\usepackage{bbold}
\usepackage{enumitem}
\usepackage{mathtools}

\DeclarePairedDelimiter\ceil{\lceil}{\rceil}
\DeclarePairedDelimiter\floor{\lfloor}{\rfloor}

\usepackage{fancyhdr}
\usepackage{mathrsfs}
\usepackage{xcolor}

\usepackage{listings}
\usepackage{hyperref}
\usepackage[T1]{fontenc}
\usepackage{tabularx}

\usepackage{mathpazo}
\usepackage{xcolor}
\usepackage{float}

\usepackage{graphicx}
\usepackage{subcaption}
\usepackage[export]{adjustbox}
\usepackage{wrapfig}


\usepackage{tikz,lipsum,lmodern}
\usetikzlibrary{calc}
\usetikzlibrary{arrows}
\usepackage{pgfplots}
\usepackage{graphicx}
\makeatletter
\usepackage[most]{tcolorbox}

\setlength{\parindent}{0pt} 
\theoremstyle{plain}
\newtheorem*{theorem}{Theorem}

\theoremstyle{remark}
\newtheorem*{solution}{Solution}

\theoremstyle{plain}
\newtheorem*{claim}{Claim}

\newcolumntype{P}[1]{>{\centering\arraybackslash}p{#1}}
\pagestyle{fancy}
% \newcommand{\ehx}[]{} set new commands
\newcommand{\bd}{\textbf}
\newcommand\course{CSE 3500}

\newcommand{\fa}{\forall}

\newcommand{\nn}{\mathbb{N}}
\newcommand{\z}{\mathbb{Z}}
\newcommand{\rn}{\mathbb{R}}
\newcommand{\stopIndent}{\noindent\underline\bd}


\newcommand*{\rom}[1]{\expandafter\@slowromancap\romannumeral #1@}
\headheight 35pt
\rhead{Edward Song \\ \today \\ Math 138}
\title{Math 135}
\author{Edward Song}

\headsep 1.8em
\lfoot{}
\cfoot{}
\rfoot{\small\thepage}

\newtcbtheorem[auto counter,number within=section]{theo}%
  {Theorem}{fonttitle=\bfseries\upshape, fontupper=\slshape,
     arc=0mm, colback=blue!5!white,colframe=blue!75!black}{theorem}

\begin{document}

\section{Integration}


% \begin{tcolorbox}[colback=red!5!white,colframe=red!75!black,title=Recall]
   
% \end{tcolorbox}   

\subsection*{1.5  The Fundamental Theorem of Calculus }

This theorem provides a connection between differential and integral calculus 
and is expressed in 2 parts.


\begin{tcolorbox}[colback=green!5!white,colframe=green!75!black,title=Definition: FTC Part 1]
    Assume $f$ is continuous on an open interval $I$ containing a point $a$. Let 
    \[G(x) = \int_a^x \,f(t)\,dt\]

    Then $G(x)$ is differentiable at each $x\in I$ and \[G'(x) = f(x)\]
    
    Equivalently, \[G'(x) = \frac{d}{dx}\int_a^x\,f(t)dt\]
\end{tcolorbox}   

\begin{proof}
    By definition:
    \begin{align*}
        g'(x) &= \lim_{h\to 0}\frac{g(x+h)-g(x)}{h} \\
        &=\lim_{h\to 0}\frac{1}{h}(\int_a^{x+h}f(t)dt-\int_a^a\,f(t)dt) \\
        &=\lim_{h\to 0}\frac{1}{h}\int_x^{x+h}f(t)dt \\
        &=\lim_{h\to 0}f(c)\,\text{for } c\in[x,x+h] \\
        &=f(x) \, \text{ since $f$ is continuous and $c\to x$ as $h\to 0$}
    \end{align*}
\end{proof}

\begin{tcolorbox}[colback=magenta!5!white,colframe=magenta!75!black,title=Example]
    \bd{Problem:} \\For $g(x) = \int_0^x\frac{dt}{\sqrt{1+t^4}}$ for $x>0$, find $g'(2)$. \\

    \bd{Solution:} \\
    Let $f(t) = \frac{1}{\sqrt{1+t^4}}$, then $f(t)$ is continuous for all $t \rightarrow$ FTC 1 applies. Thus,
    $g'(x) = \frac{1}{\sqrt{1+x^4}}$ for $x>0$ by FTC1. \[g'(2) = \frac{1}{17}\]
\end{tcolorbox}   

\begin{tcolorbox}[colback=magenta!5!white,colframe=magenta!75!black,title=Example]
    \bd{Problem:} \\Let $p(x) = \int_1^{x^2}\frac{1}{t}e^{-t}dt$, find $p'(x)$.

    \bd{Solution:} \\
    Define $f(t) = \frac{1}{t}e^{-t}$, then $f(t)$ is continuous for all $t\neq 0$. FTC1 applies for all $t\neq 0$.\\

    Let $u=x^2$, then 
    \begin{align*}
        p(u) - \int_1^u\frac{1}{t}e^{-t}dt\rightarrow \frac{dp}{dx} = p'(x) &= \frac{dp}{du}\cdot \frac{du}{dx} \\
        &=(\frac{1}{u}e^{-u})(2x) \\
        &=\frac{2}{x}e^{-x^2}\,\text{ for $x\neq 0$}
    \end{align*}
\end{tcolorbox}   

\begin{tcolorbox}[colback=magenta!5!white,colframe=magenta!75!black,title=Example]
    \bd{Problem:} \\For $H(x) = \int_{x^2}^{e^x}\cos(t^2)dt$ find $H'(x)$

    \bd{Solution:} \\
    Define $f(t) = \cos(t^2)$, then $f(t)$ is continuous for all $t$. \\

    Let $u=x^2$ and $v=e^x$, then

    \begin{align*}
        H = \int_u^v\,f(t)dt &= \int_u^a\,f(t)dt + \int_a^v\,f(t)dt \\
        &=-\int_a^u\,f(t)dt + \int_a^v\,f(t)dt
    \end{align*}

    Next, apply the Chain rule and FTC1 to get:
    \begin{align*}
        H'(x) = \frac{dH}{dx} &= -f(u)\frac{du}{dx} + f(v)\frac{dv}{dx} \\
        &= e^x\cos(e^{2x})-2x\cos(x^4) 
    \end{align*}

    This leads to the generalized version of the FTC1.
\end{tcolorbox}   

\begin{tcolorbox}[colback=green!5!white,colframe=green!75!black,title=Definition: FTC1 - Extended Version]
    Assume that $f$ is continuous and $g+h$ are differentiable. \\

    Let \[H(x) = \int_{g(x)}^{h(x)}f(t)dt\]

    Then $H(x)$ is differentiable and $H'(x) = f(h(x))h'(x)-f(g(x))g'(x)$.
\end{tcolorbox}   


\begin{tcolorbox}[colback=green!5!white,colframe=green!75!black,title=Definition: FTC2]
    If $f$ is continuous on $[a,b]$, then \[\int_a^b\,f(x)dx = F(b) - F(a)\] where $F'(x) = f(x)$
\end{tcolorbox}   

\bd{Notes:}

\begin{enumerate}
    \item A function $F$ is called an antiderivative of $f$ on an interval $I$ if $F'(x)=f(x)$ for all $x\in I$.
    \item If $F$ is an antiderivative of $f$, then so is $F+c$ for all $c\in \rn$.
    \item We define the evaluation symbol as: \[F(x)|_a^b = F(b)-F(a)\]
    \item Using FTC2 is more practical for evaluataing integrals than using Riemann Sums. However, FTC2 assumes that the antiderivative of $f$ is known. We will learn some techniques of integration to help us determine the antiderivatives.
\end{enumerate}

\end{document}