\documentclass[11pt]{article}

\usepackage{amsmath, amsfonts, amssymb, amsthm}
\usepackage{braket}
\usepackage{fullpage}
\usepackage[top=2cm, bottom=4.5cm, left=2.5cm, right=2.5cm]{geometry}
\usepackage{bbold}
\usepackage{enumitem}
\usepackage{mathtools}

\DeclarePairedDelimiter\ceil{\lceil}{\rceil}
\DeclarePairedDelimiter\floor{\lfloor}{\rfloor}

\usepackage{fancyhdr}
\usepackage{mathrsfs}
\usepackage{xcolor}

\usepackage{listings}
\usepackage{hyperref}
\usepackage[T1]{fontenc}
\usepackage{tabularx}

\usepackage{mathpazo}
\usepackage{xcolor}
\usepackage{float}

\usepackage{graphicx}
\usepackage{subcaption}
\usepackage[export]{adjustbox}
\usepackage{wrapfig}


\usepackage{tikz,lipsum,lmodern}
\usetikzlibrary{calc}
\usetikzlibrary{arrows}
\usepackage{pgfplots}
\usepackage{graphicx}
\makeatletter
\usepackage[most]{tcolorbox}

\setlength{\parindent}{0pt} 
\theoremstyle{plain}
\newtheorem*{theorem}{Theorem}

\theoremstyle{remark}
\newtheorem*{solution}{Solution}

\theoremstyle{plain}
\newtheorem*{claim}{Claim}

\newcolumntype{P}[1]{>{\centering\arraybackslash}p{#1}}
\pagestyle{fancy}
% \newcommand{\ehx}[]{} set new commands
\newcommand{\bd}{\textbf}
\newcommand\course{CSE 3500}

\newcommand{\fa}{\forall}

\newcommand{\nn}{\mathbb{N}}
\newcommand{\z}{\mathbb{Z}}
\newcommand{\rn}{\mathbb{R}}
\newcommand{\stopIndent}{\noindent\underline\bd}


\newcommand*{\rom}[1]{\expandafter\@slowromancap\romannumeral #1@}
\headheight 35pt
\rhead{Edward Song \\ \today \\ Math 138}
\title{Math 135}
\author{Edward Song}

\headsep 1.8em
\lfoot{}
\cfoot{}
\rfoot{\small\thepage}

\newtcbtheorem[auto counter,number within=section]{theo}%
  {Theorem}{fonttitle=\bfseries\upshape, fontupper=\slshape,
     arc=0mm, colback=blue!5!white,colframe=blue!75!black}{theorem}

\begin{document}

\section{Integration}


% \begin{tcolorbox}[colback=red!5!white,colframe=red!75!black,title=Recall]
   
% \end{tcolorbox}   

\subsection*{1.6 The Fundamental Theorem of Calculus (Part 2)}

\subsubsection*{Indefinite Integrals}
Recall that the definite integral, $\int_a^b\,f(x)dx$, is a number, where as the \underline{indefinite integral}, $\int\,f(x)dx$, is a function
and is the antiderivative of $f(x)$.

\begin{tcolorbox}[colback=red!5!white,colframe=red!75!black,title=Definition: Recall]
   FTC2: $\int_a^b\,f(x)dx = F(x)|_a^b = F(b)-F(a)$ where $F'(x) = f(x)$. \\

   In order to apply FTC2 or evaluate indefinite integrals, it's important to remember the antiderivative of 
   elementary functions.
\end{tcolorbox}   

Note that there are many functions for which antiderivatives cannot be expressed in terms of elementary functions, such as
\[\int e^{-x^2}, \int \sqrt{1+x^3}dx, \int\sin(x^2)dx\]

\begin{tcolorbox}[colback=magenta!5!white,colframe=magenta!75!black,title=Problem 1 ]
    \begin{enumerate} [label=(\alph*)]
        \item $\int\frac{dx}{a+x} = \ln|x+a|+c$
        \item $\int\frac{dx}{a+bx} = \frac{1}{b}\ln|a+bx|+c$
        \item $\int(a+bx)^ndx = \frac{1}{b(n+1)}(a+bx)^{n+1}+C, (n\neq 1)$
        \item $\int\cos(a+bx)dx=\frac{1}{b}\sin(a+bx) +C$
        \item $\int e^{a+bx} = \frac{1}{b}e^{a+bx} +C$
    \end{enumerate}
\end{tcolorbox}   

\begin{tcolorbox}[colback=magenta!5!white,colframe=magenta!75!black,title=Problem 2 ]
    Evaluate $\int\cos^2xdx$:\\

    Even powers of $\sin x$ or $\cos x$ can be integrated by using trig identities. In this case, use \[\cos^2x = \frac{1}{2}(1+2\cos 2x)dx = \frac{1}{2}(x+\frac{1}{2}\sin 2x)+C\]

    \bd{\underline{Note:}} $\int\cos^2xdx \neq \frac{1}{3}\cos^3x+C$ and is easily shown by differentiation:
    \[\frac{d}{dx}(\frac{1}{3}\cos^3x+C) = -\sin\cos^2x \neq \cos^2x\]
\end{tcolorbox}   

\begin{tcolorbox}[colback=magenta!5!white,colframe=magenta!75!black,title=Problem 3 ]
    Evaluate $\int\frac{x}{x+1}dx$:\\

    When integrating rational functions always perform long divsion when the degree of the numerator is greater than 
    or equal to the degree of the denominator. Here, \[\frac{x}{x+1} = 1-\frac{1}{x+1}\]\[\int\frac{x}{x+1}dx = \int(1-\frac{1}{1+x})dx=x-\ln|x+1|+C\]
\end{tcolorbox}   

\begin{tcolorbox}[colback=magenta!5!white,colframe=magenta!75!black,title=Problem 4]
    Evaluate $\int\frac{2+e^{-x}}{e^x}dx$ \\

    Sometimes it's helpful to manipulate/rewrite the integrand. Here, $\frac{2+e^{-x}}{2^x} = 2e^{-x} + e^{-2x}$.
    \begin{align*}
        &\int\frac{2+e^{-x}}{e^x}dx \\ 
        =&\int(2e^{-x}+e^{-2x})dx  \\
        =&-2e^{-x}-\frac{1}{2}e^{-2x}+C 
    \end{align*}
\end{tcolorbox}   

\begin{tcolorbox}[colback=magenta!5!white,colframe=magenta!75!black,title=Problem 5]
    Evaluate $\int_0^2|x^2-3x+2|dx$\\

    \begin{align*}
        \int_0^2|x^2-3x+2|dx &=\int_0^1(x^2-3x+2)dx+\int_1^2(-x^2+3x-2)dx \\
        &=(\frac{1}{3}x^3-\frac{3}{2}x^2+2x)\Big|_0^1+(-\frac{1}{3}x^3+\frac{3}{2}x^2-2x)\Big|_1^2 \\
        &=(\frac{1}{3}-\frac{3}{2}+2)-0+(-\frac{8}{3}+6-4)-(-\frac{1}{3}+\frac{3}{2}-2) \\
        &= 1
    \end{align*}
\end{tcolorbox}   



\end{document}