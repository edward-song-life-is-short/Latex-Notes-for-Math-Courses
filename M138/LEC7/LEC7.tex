\documentclass[11pt]{article}

\usepackage{amsmath, amsfonts, amssymb, amsthm}
\usepackage{braket}
\usepackage{fullpage}
\usepackage[top=2cm, bottom=4.5cm, left=2.5cm, right=2.5cm]{geometry}
\usepackage{bbold}
\usepackage{enumitem}
\usepackage{mathtools}

\DeclarePairedDelimiter\ceil{\lceil}{\rceil}
\DeclarePairedDelimiter\floor{\lfloor}{\rfloor}

\usepackage{fancyhdr}
\usepackage{mathrsfs}
\usepackage{xcolor}

\usepackage{listings}
\usepackage{hyperref}
\usepackage[T1]{fontenc}
\usepackage{tabularx}

\usepackage{mathpazo}
\usepackage{xcolor}
\usepackage{float}

\usepackage{graphicx}
\usepackage{subcaption}
\usepackage[export]{adjustbox}
\usepackage{wrapfig}


\usepackage{tikz,lipsum,lmodern}
\usetikzlibrary{calc}
\usetikzlibrary{arrows}
\usepackage{pgfplots}
\usepackage{graphicx}
\makeatletter
\usepackage[most]{tcolorbox}

\setlength{\parindent}{0pt} 
\theoremstyle{plain}
\newtheorem*{theorem}{Theorem}

\theoremstyle{remark}
\newtheorem*{solution}{Solution}

\theoremstyle{plain}
\newtheorem*{claim}{Claim}

\newcolumntype{P}[1]{>{\centering\arraybackslash}p{#1}}
\pagestyle{fancy}
% \newcommand{\ehx}[]{} set new commands
\newcommand{\bd}{\textbf}
\newcommand\course{CSE 3500}

\newcommand{\fa}{\forall}

\newcommand{\nn}{\mathbb{N}}
\newcommand{\z}{\mathbb{Z}}
\newcommand{\rn}{\mathbb{R}}
\newcommand{\stopIndent}{\noindent\underline\bd}


\newcommand*{\rom}[1]{\expandafter\@slowromancap\romannumeral #1@}
\headheight 35pt
\rhead{Edward Song \\ \today \\ Math 138}
\title{Math 135}
\author{Edward Song}

\headsep 1.8em
\lfoot{}
\cfoot{}
\rfoot{\small\thepage}

\newtcbtheorem[auto counter,number within=section]{theo}%
  {Theorem}{fonttitle=\bfseries\upshape, fontupper=\slshape,
     arc=0mm, colback=blue!5!white,colframe=blue!75!black}{theorem}

\begin{document}

\subsection*{2.2 Techniques of Integration}
The integration technique corresponds to the integral version of the product rule.\\

In integral form this becomes:\[\int udv=uv-\int vdu\]
Note that $\frac{dv}{dx}=dv, \frac{du}{dx}dx=du$ and the constant is ignored.
% \begin{tcolorbox}[colback=red!5!white,colframe=red!75!black,title=Recall]

% \end{tcolorbox}   
In this method, we break up the integral into a product of two parts: $u$ and $dv$, where dv is easily integrated and the integral
$\int vdu$ on the RHS is simpler than the original integral $\int u dv$.

\begin{tcolorbox}[colback=magenta!5!white,colframe=magenta!75!black,title=Example]
   \begin{enumerate}
       \item $\int xe^xdx$. \\
       
       Let $u=x, dv=e^xdx$, then $du=dx, v=e^x$
       \[\int xe^xdx = xe^x-\int e^xdx=xe^x-e^x+C\]
       \item Evaluate $\int x^2\sin xdx$. Let $u=x^2, dv=\sin x dx$, then $du=2xdx, v=-\cos x$,
       \[\int x^2\sin xdx = -x^2\cos x+2\int x\cos xdx\]
       We can apply IBP again which yields: $u=x, dv=\cos x dx$ then $du=dx, v=\sin x$.

        \[\int x\cos xdx=x\sin x-\int \sin x dx=x\sin x+\cos x+C\]Hence,
       \begin{align*}
           \int x^2\sin x dx&=-x^2\cos x+2[x\sin x+\cos x+C]\\
           &=-x^2\cos x+2x\sin x+2\cos x+K, K=2C
       \end{align*}
      
       to consider two cases: $\tan\theta>0,\tan\theta<0$.
   \end{enumerate}
\end{tcolorbox}   
\begin{tcolorbox}[colback=red!5!white,colframe=red!75!black,title=Remark ]
        For integrands of the form $x^n \{\sin(ax), \cos(ax), e^{ax}\}$ with $n$ a
        positive number, choose $u=x^n+dv=\{\sin(ax), \cos(ax), e^{ax}\}dx$
\end{tcolorbox}   

\begin{tcolorbox}[colback=magenta!5!white,colframe=magenta!75!black,title=Example]
    \begin{enumerate}
        \item $\int\ln xdx$. Let $u=\ln x, dv=dx$, then $du=\frac{dx}{x}, v=x$.
        \[\ln xdx =x\ln x-\int dx = x\ln x-x+C\]

        \bd{Remark:} For integrals involving $\ln x$, try $u=\ln x$.

        \item Evaluate $\int x\tan^{-1}xdx$. Let $u=\tan^{-1}x, dv=xdx$, then $du=\frac{dx}{1+x^2}, v=\frac{1}{2}x^2$.
        \begin{align*}
            \int x \tan^{-1}xdx&=\frac{1}{2}x^2\tan_{-1}x-\frac{1}{2}\int\frac{x^2dx}{1+x^2}\\
            &=\frac{1}{2}x^2\tan^{-1}-\frac{1}{2}\int[1-\frac{1}{1+x^2}]dx \,\text{by long division}\\
            &=\frac{1}{2}x^2\tan^{-1}-\frac{1}{2}x+\frac{1}{2}\tan^{-1}x+C
        \end{align*}

        \item Evaluate $\int_1^ex\ln xdx$. Let $u=\ln x, dv=xdx$, then $du=\frac{dx}{x}, v=\frac{1}{2}x^2$.
        \begin{align*}
            \int_1^e\ln xdx&=\frac{1}{2}x^2\ln x\Big|_1^e-\frac{1}{2}\int_1^exdx\\
            &=\frac{1}{2}e^2-\frac{1}{4}x^2|_1^{e^1}=\frac{1}{4}(e^2+1)
        \end{align*}

        \item Evaluate $\int e^{ax}\cos(bx)dx$. Can either pick $u=e^{ax}+dx=\cos(bx)dx$ or $u=\cos(bx)+dv=e^{ax}dx$\\
        Let's use $u=e^{ax}, dv=\cos(bx)dx$, then $du=ae^{ax}dx, v=\frac{1}{b}\sin(bx)$.
        \[\int e^{ax}\cos(bx)dx=\frac{1}{b}e^{ax}\sin(bx)-\frac{a}{b}\int e^{ax}\sin(bx)dx\]
        We apply IBP again so: $u=e^{ax}, dv=\sin(bx)dx$, then $du=ae^{ax}dx, v=-\frac{1}{b}\cos(bx)$.
        \[\int e^{ax}\cos(bx)dx=\frac{1}{b}e^{ax}\sin(bx)-\frac{a}{b}[-\frac{1}{b}e^{ax}\cos(bx)+\frac{a}{b}\int e^{ax}\cos(bx)dx]\]
        We can combine both these integrals:
        
        \[\rightarrow (1+\frac{a^2}{b^2}\int e^{ax}\cos(bx)dx=\frac{1}{b}e^{ax}\sin(bx)+\frac{a}{b^2}e^{ax}\cos(bx)\]
        
        \begin{align*}
            \rightarrow \int e^{ax}\cos(bx)dx&=\frac{\frac{1}{b}e^{ax}(\sin(bx)+\frac{a}{b}\cos(bx))}{(1+\frac{a^2}{b^2})}+C\\
            &=\frac{e^{ax}(b\sin(bx)+a\cos(bx))}{a^2+b^2}+C
        \end{align*}
    \end{enumerate}
 \end{tcolorbox}   



\end{document}